\documentclass[a4paper,ngerman,12pt]{scrreprt}

\usepackage[ngerman]{babel}
%Deutsche Anführungszeichen
\usepackage[utf8]{inputenc}
\usepackage[autostyle=true,german=quotes]{csquotes}
\usepackage{amsmath}
\usepackage{graphicx}
\usepackage[colorinlistoftodos]{todonotes}
\usepackage{eurosym}
\usepackage{lmodern}
\usepackage{hyperref}
%pdf einbinden
\usepackage{pdfpages}

%Bibliographie einbinden
\usepackage[bibstyle=numeric,citestyle=authoryear,backend=biber]{biblatex}
\bibliography{mobileapps.bib}

%Glossar einbinden
\usepackage[acronym,acronymlists={main}]{glossaries}
\makeglossaries
	\setacronymstyle{footnote}
	\newacronym{LMS}{LMS}{Learning Management System}
	\newacronym{SCORM}{SCORM}{Sharable Content Object Reference Model}
	\newacronym{LOM}{LOM}{Learning Objects Metadata}
	\newacronym{SSL}{SSL}{SSL-Verschlüsselung -- Secure Sockets Layer}
	\newacronym{LAMPP}{LAMPP}{Webserver bestehend aus: Linux, Apache, MySQL, PHP und Perl}
	\newacronym{responsive Design}{Responsive Design}{Gestalterisches und technisches Paradigma zur Erstellung von Internetseiten, so dass diese Seiten auf die jeweiligen Geräte des Nutzers (speziell Smartphones und Tablets) reagieren können}
%Für maketitle
\title{Mobile Applications}
\author{
	Oliver Friedrich
	\and Marko Klepatz
	\and Klaus Steinhauer
    }

\date{\today}
\begin{document}
%opening
\begin{titlepage}
	\begin{center}
	
	\textsc{\large Deutsche Telekom Hochschule für Telekommunikation Leipzig}
	\\[2cm]
	{\Large Beleg zum Modul Mobile Applications}
	\\[1cm]
		\begin{minipage}{0.2\textwidth}
		\begin{flushleft}
		{\footnotesize Thema}
		\end{flushleft}
		\end{minipage}
		 \hfill
		\begin{minipage}{0.75\textwidth}
		\begin{flushleft}
		\textbf{{\Large NFC APP .... //TODO: genauer titel}}
		\end{flushleft}
		 \end{minipage}
		\\[2cm]
		\begin{minipage}{0.2\textwidth}
		\begin{flushleft}
		{\footnotesize vorgelegt von}
		\end{flushleft}
		\end{minipage}
		 \hfill
		\begin{minipage}{0.75\textwidth}
		\begin{flushleft}
		{\normalsize
			Oliver Friedrich
			\\Marko Klepatz
			\\Klaus Steinhauer}
		\end{flushleft}
		 \end{minipage}
		 \\[2cm]
		 \begin{minipage}{0.2\textwidth}
		 \begin{flushleft}
		 {\footnotesize Betreuer}
		 \end{flushleft}
		 \end{minipage}
	  \hfill
	   \begin{minipage}{0.75\textwidth}
	  		 \begin{flushleft}
	  		 {\normalsize
	  		 		Prof. Dr. rer. nat. Ulf Schemmert
	  		 }
	  		 \end{flushleft}
	  		  \end{minipage}
	  		  \vfill
	 \renewcommand*{\dictumwidth}{.6\textwidth}
	 \dictum[Klaus]{
	 ''Einmal mit Profis arbeiten...''
	 } 	
	
	%\maketitle
	\end{center}
\end{titlepage}

\begingroup
\renewcommand*{\chapterpagestyle}{empty}
\pagestyle{empty}
\tableofcontents
\clearpage
\endgroup

\chapter{Einleitung}


\chapter{Struktur}
 
\begin{quote}
	''Haha! du Schuft!''
\end{quote}

\chapter{Node Rocks!}

\section{Motivation}

\begin{itemize}
\item Item Sample
\item All your Items are belong to us!
\end{itemize}

\section{Entwicklung}




\chapter{Fazit}
Weil wir nix können.

\begingroup
\let\clearpage\relax
\chapter{Anhang}
\printbibliography
%\printglossaries
\endgroup
\vfill

\section*{Selbstständigkeitserklärung}
Hiermit erklären wir, dass wir die von uns an der Deutschen Telekom Hochschule für Telekommunikation Leipzig eingereichte Belegarbeit zum Thema\\[1cm]
\centering \\[1cm]
selbständig verfasst und keine anderen als die angegebenen Hilfsmittel benutzt haben.
\\[1.5cm]
\flushleft Leipzig, den \today
\end{document}
